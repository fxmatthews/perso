%% start of file `template_en.tex'.
%% Copyright 2006-1008 Xavier Danaux (xdanaux@gmail.com).
% 
% This work may be distributed and/or modified under the
% conditions of the LaTeX Project Public License version 1.3c,
% available at http://www.latex-project.org/lppl/.

\documentclass[10pt,a4paper]{moderncv}
\moderncvtheme[blue]{classic}   

\usepackage[utf8]{inputenc}
\usepackage[T1]{fontenc}
\usepackage[francais]{babel}
\usepackage{graphicx}
\usepackage{amsmath}
\usepackage{amsfonts}
\usepackage{amssymb}

\usepackage[scale=0.9]{geometry}

 \setlength{\hintscolumnwidth}{2.7cm}

\AtBeginDocument{\recomputelengths}

\firstname{François-Xavier}
\familyname{\\Matthews}
\title{Recherche de stage ingénieur}
\address{A206, 21 avenue de la mare aux daims}{76800 Saint-Etienne-Du-Rouvray}
\extrainfo{21 ans (24/02/1991)\\
  Nationalité française\\Permis B} 

\begin{document}
\maketitle

\section{Formation}

\cventry{2008-2014}{Élève ingénieur}{}{Institut National des Sciences Appliquées (INSA) de Rouen (76)}{}{Département Architecture des Systèmes d'Information (2011-2014)}

\section{Projets universitaires}
\cventry{2013}{Projet INSA Certifié (ISO 9001)}{Développement d'un logiciel d'analyse et de stockage de méta-données en Python suivant le pattern MVC pour la société A2iA. Utilisation de Django. Gestion de la qualité du projet}{}{}{}
\cventry{2013}{Data mining}{Implémentations de différents algorithmes de classification (Réseaux de neurones, SVM) en matlab}{}{}{}

\section{Expériences professionnelles}
\cventry{Juin-Août 2013}{Stage de spécialité}{Développement d'une application web J2EE dans un contexte de création d'entreprise. Utilisation du framework Spring et de ses modules (MVC, security, ldap, social...). Utilisation des apis de facebook et linkedin.}{Dynamease}{Vernon (27200)}{}
\cventry{Juillet-Août 2012}{Stage technicien}{Développement d'une application de tests via Java, Selenium Driver et JUnit pour une application web}{Bebook}{Paris (75)}{}
\cventry{été 2011}{Préparation Militaire Découverte}{}{Forteresse du Mont Valérien, Suresnes (92)}{}{8$^{ème}$ régiment de transmission de l'Armée de Terre}{}


\section{Cursus}
\cvline{\textbf{Technologie}}{Systèmes d'exploitation (Linux), base de données}
\cvline{\textbf{Mathématiques}}{Analyse numérique, traitement du signal, automatique, statistiques, data mining, apprentissage par contexte, traitement de l'image.}
\cvline{\textbf{Informatique}}{Algorithmique, UML, documents, Java, C, SQL}

\section{Technologies utilisées}
\cvcomputer{\textbf{Outils}}{Eclipse, Matlab, SVN, GIT}{\textbf{Bureautique}}{\LaTeX, openoffice (C2I)}
\cvcomputer{\textbf{OS}}{GNU/Linux, Emacs, Windows}{\textbf{Modélisation}}{Dia, Bouml}
\cvcomputer{\textbf{Langages}}{Matlab, Python, Java, Javascript, UML}{\textbf{Autre}}{Unity3D}

\section{Langues étrangères}
\cvline{\textbf{Anglais}}{Courant (975/990 au TOEIC)}
\cvline{\textbf{Allemand}}{Niveau scolaire (B2)}

\section{Projets personnels}
\cvline{\textbf{Jeux vidéos}}{Organisation de tournois de jeux vidéo à l'INSA}
\cvline{\textbf{Musique}}{Création d'une chorale étudiante.}
\end{document}